\chapter{Zaključak i budući rad}
	
	% \textbf{\textit{dio 2. revizije}}\\
	
	% 	\textit{U ovom poglavlju potrebno je napisati osvrt na vrijeme izrade projektnog zadatka, koji su tehnički izazovi prepoznati, jesu li riješeni ili kako bi mogli biti riješeni, koja su znanja stečena pri izradi projekta, koja bi znanja bila posebno potrebna za brže i kvalitetnije ostvarenje projekta i koje bi bile perspektive za nastavak rada u projektnoj grupi.}
	
	% 	\textit{Potrebno je točno popisati funkcionalnosti koje nisu implementirane u ostvarenoj aplikaciji.}

	{Zadatak naše grupe bio je razvoj web aplikacije koja bi služila neprofitnim organizacijama kao baza podataka projekata i kompanija s kojima se surađivalo na istim. Nakon 12 tjedana rada u timu i razvoja, približno smo ostvarili zadani cilj. Sama provedba projekta bila je kroz dvije faze.}

	{Prva faza projekta uključivala je okupljanje tima za razvoj aplikacije, dodjelu projektnog zadatka i intenzivan rad na dokumentiranju zahtjeva. Kvalitetna provedba prve faze uvelike je olakšala daljnji rad pri realizaciji osmišljenog sustava. Izrađeni obrasci i dijagrami (obrasci uporabe, sekvencijski dijagrami, model baze podataka, dijagram razreda) bili su od pomoći podtimovima zaduženima za razvoj backenda i frontenda. Izrada vizualnih prikaza idejnih rješenja problemskog zadatka uštedjela je mnogo vremena u drugom ciklusu kada su članovi tima nailazili na nedoumice oko implementacije rješenja.}

	{Druga faza projekta, iako nešto kraća od prve, bila je puno intenzivnija po pitanju samostalnog rada članova. Manjak iskustva članova u izradi sličnih implementacijskih rješenja primorao je članove na samostalno učenje odabranih alata i programskih jezika kako bi ispunili dogovorene ciljeve. Osim realizacije rješenja, u drugoj fazi je bilo potrebno dokumentirati ostale UML dijagrame i izraditi popratnu dokumentaciju kako bi budući korisnici mogli lakše koristiti ili vršiti preinake na sustavu. Dobro izrađen kostur projekta uštedio nam je mnogo vremena prilikom izrade aplikacije te smo izbjegli moguće pogreške u izradi koje bi bile vremenski skupe za ispravljanje u daljnjoj fazi projekta. Komunikacija među članovima tima bila je putem WhatsAppa čime smo postigli informiranost svih članova grupe o napretku projekta. Moguće proširenje postojeće inačice sustava je izrada mobilne aplikacije čime bi se cilj projektnog zadatka bio ostvaren u većoj mjeri no s web aplikacijom. Sudjelovanje na ovakvom projektu bilo je vrijedno iskustvo svim članovima tima jer smo kroz intenzivnih nekoliko tjedana rada iskusili zajednički rad na istom projektu. Također, osjetili smo važnost dobre vremenske organiziranosti i koordiniranosti između članova tima.}
	
	{Zadovoljni smo postignutim bez obzira na golemi prostor za usavršavanje izrađene aplikacije što je posljedica neiskustva na takvim i sličnim projektima.}
	
	\eject