\chapter{Implementacija i korisničko sučelje}
		
		\section{Korištene tehnologije i alati}
		
			% \textbf{\textit{dio 2. revizije}}
			
			%  \textit{Detaljno navesti sve tehnologije i alate koji su primijenjeni pri izradi dokumentacije i aplikacije. Ukratko ih opisati, te navesti njihovo značenje i mjesto primjene. Za svaki navedeni alat i tehnologiju je potrebno \textbf{navesti internet poveznicu} gdje se mogu preuzeti ili više saznati o njima}.

            {Većinski dio komunikacije održan je putem WhatsApp\footnotemark{}\footnotetext{https://www.whatsapp.com/} komunikacijskog kanala, dok je drugi dio održan ili uživo ili preko platforme Discord\footnotemark{}\footnotetext{https://discord.com/}. Prilikom izrade dijagrama korišten je alat Astah\footnotemark{}\footnotetext{https://astah.net/}, dok je za pisanje dokumentacije korišten online editor Overleaf\footnotemark{}\footnotetext{https://www.overleaf.com/}.}\vspace{0.3cm}

            {U svrhu verzioniranja koda od strane svih članova grupe korišten je alat Git\footnotemark{}\footnotetext{https://git-scm.com/}, dok je kao udaljeni repozitorij korištena platforma GitLab\footnotemark{}\footnotetext{https://gitlab.com/}.}\vspace{0.2cm}
            
            {Za razvoj Backend dijela aplikacije korišten je Java JDK\footnotemark{}\footnotetext{https://www.oracle.com/java/technologies/downloads/}, gradle\footnotemark{}\footnotetext{https://gradle.org/} i Intellij Idea\footnotemark{}\footnotetext{https://www.jetbrains.com/idea/} razvojno okruženje za razvoj Spring boot\footnotemark{}\footnotetext{https://spring.io/projects/spring-boot} aplikacije. Za Frontend dio aplikacije korišten je Node.js\footnotemark{}\footnotetext{https://nodejs.org/en/} kao program za pokretanje i React\footnotemark{}\footnotetext{https://reactjs.org/}, također poznat kao React.js ili ReactJS, kao Javascript\footnotemark{}\footnotetext{https://www.javascript.com/} bibiloteka za izradu sučelja. Kao razvojno okruženje za React korišten je VS Code\footnotemark{}\footnotetext{https://code.visualstudio.com/} IDE.}\vspace{0.3cm}
            
            {Za upravljanje bazom podataka korišten je PostgreSQL\footnotemark{}\footnotetext{https://www.postgresql.org/} instaliran na Ubuntu serveru\footnotemark{}\footnotetext{https://ubuntu.com/download/server} verzije 20.04.}
			
			\eject 
		
	
		\section{Ispitivanje programskog rješenja}
			
			\subsection{Ispitivanje komponenti}
			{Svi endpointovi na backend-u su ručno testirani, a ispitivanje jedinica (engl. unit testing) proveli smo nad osnovnim funkcionalnostima servisnog sloja koristeći biblioteku junit. Prilikom testiranja pojedinog razreda, servise i repozitorije čije metode taj razred poziva smo \textit{mock-ali} koristeći biblioteku mockito, pa smo samo u svakoj test metodi zadali što će pojedina pozvana metoda vraćati.}
			
			\textbf{1. UserServiceTests\\}
			{U UserServiceTests testnoj klasi ispitali smo funkcionalnost dodavanja novog korisnika u bazu podataka.\\
			Naredbom \textit{Mockito.when(userRepository.save(Mockito.any())).thenReturn(test);} smo rekli mockanom userRepository-ju kako da se ponaša prilikom poziva save metode.}
			\begin{figure}[H]
				\includegraphics[width=\textwidth]{UnitTests/addUserTest}
				\centering
				\caption{Add user test}
				\label{fig:addUserTest}
			\end{figure}
			\begin{figure}[H]
				\includegraphics[width=\textwidth]{UnitTests/addUserTestResult}
				\centering
				\caption{Add user test result}
				\label{fig:addUserTestResult}
			\end{figure}
			
			\textbf{2. ProjectServiceTests\\}
			{U ProjectServiceTests klasi smo ispitali funkcionalnosti dohvata projekta iz baze podataka te dodavanje FR team membera na projekt.}
			\begin{figure}[H]
				\includegraphics[width=\textwidth]{UnitTests/projectFindById}
				\centering
				\caption{Find project by Id test}
				\label{fig:projectFindByIdTest}
			\end{figure}
			\begin{figure}[H]
				\includegraphics[width=\textwidth]{UnitTests/projectFindByIdResult}
				\centering
				\caption{Find project by Id test result}
				\label{fig:projectFindByIdResult}
			\end{figure}

			\begin{figure}[H]
				\includegraphics[width=\textwidth]{UnitTests/projectAddFrTeamMember}
				\centering
				\caption{Add FR team member to project test}
				\label{fig:addFrTeamMemberTest}
			\end{figure}
			\begin{figure}[H]
				\includegraphics[width=\textwidth]{UnitTests/projectAddFrTeamMemberResult}
				\centering
				\caption{Add FR team member result}
				\label{fig:addFrTeamMemberResult}
			\end{figure}

			\textbf{3. CompanyServiceTests\\}
			{U CompanyServiceTests klasi smo ispitali sljedeće funkcionalnosti:\\
				Kreiranje nove kompanije\\
				Dohvaćanje svih kompanije\\
				Bacanje EntityNotFoundException-a prilikom pokušaja brisanja nepostojeće kompanije\\
				Bacanje AuthenticationException-a prilikom pokušaja dohvata podataka od kompanije od strane Usera koji je samo Observer}
			
			\begin{figure}[H]
				\includegraphics[width=\textwidth]{UnitTests/createCompany}
				\centering
				\caption{Create company test}
				\label{fig:createCompanyTest}
			\end{figure}
			\begin{figure}[H]
				\includegraphics[width=\textwidth]{UnitTests/createCompanyResult}
				\centering
				\caption{Create company test result}
				\label{fig:createCompanyResult}
			\end{figure}

			\begin{figure}[H]
				\includegraphics[width=\textwidth]{UnitTests/getAllCompanies}
				\centering
				\caption{Get all companies test}
				\label{fig:getAllCompaniesTest}
			\end{figure}
			\begin{figure}[H]
				\includegraphics[width=\textwidth]{UnitTests/getAllCompaniesResult}
				\centering
				\caption{Get all companies test result}
				\label{fig:getAllCompaniesResult}
			\end{figure}

			\begin{figure}[H]
				\includegraphics[width=\textwidth]{UnitTests/deleteCompanDoesntExist}
				\centering
				\caption{Delete company which does not exist test}
				\label{fig:deleteCompanyTest}
			\end{figure}
			\begin{figure}[H]
				\includegraphics[width=\textwidth]{UnitTests/deleteCompanDoesntExistResult}
				\centering
				\caption{Delete company which does not exist tes result}
				\label{fig:deleteCompanyResult}
			\end{figure}

			\begin{figure}[H]
				\includegraphics[width=\textwidth]{UnitTests/getCompanyObserver}
				\centering
				\caption{Get company info by observer test}
				\label{fig:getCompanyObserver}
			\end{figure}
			\begin{figure}[H]
				\includegraphics[width=\textwidth]{UnitTests/getCompanyObserverResult}
				\centering
				\caption{Get company info by observer test result}
				\label{fig:getCompanyObserverResult}
			\end{figure}
			
			\textbf{4. CollaborationServiceTests\\}
			{U CollaborationServiceTests testnoj klasi smo ispitali funkcionalnost dodavanja suradnje u bazu podataka.}

			\begin{figure}[H]
				\includegraphics[width=\textwidth]{UnitTests/addCollaboration}
				\centering
				\caption{Add collaboration test}
				\label{fig:addCollaboration}
			\end{figure}
			\begin{figure}[H]
				\includegraphics[width=\textwidth]{UnitTests/addCollaborationResult}
				\centering
				\caption{Add collaboration test result}
				\label{fig:addCollaborationResult}
			\end{figure}

			% \subsection{Ispitivanje sustava}
			
			%  \textit{Potrebno je provesti i opisati ispitivanje sustava koristeći radni okvir Selenium\footnote{\url{https://www.seleniumhq.org/}}. Razraditi \textbf{minimalno 4 ispitna slučaja} u kojima će se ispitati redovni slučajevi, rubni uvjeti te poziv funkcionalnosti koja nije implementirana/izaziva pogrešku kako bi se vidjelo na koji način sustav reagira kada nešto nije u potpunosti ostvareno. Ispitni slučaj se treba sastojati od ulaza (npr. korisničko ime i lozinka), očekivanog izlaza ili rezultata, koraka ispitivanja i dobivenog izlaza ili rezultata.\\ }
			 
			%  \textit{Izradu ispitnih slučajeva pomoću radnog okvira Selenium moguće je provesti pomoću jednog od sljedeća dva alata:}
			%  \begin{itemize}
			%  	\item \textit{dodatak za preglednik \textbf{Selenium IDE} - snimanje korisnikovih akcija radi automatskog ponavljanja ispita	}
			%  	\item \textit{\textbf{Selenium WebDriver} - podrška za pisanje ispita u jezicima Java, C\#, PHP koristeći posebno programsko sučelje.}
			%  \end{itemize}
		 % 	\textit{Detalji o korištenju alata Selenium bit će prikazani na posebnom predavanju tijekom semestra.}
			
			\eject 
		
		
		\section{Dijagram razmještaja}
			
			{ Na poslužiteljskom računalu se nalaze web poslužitelj i poslužitelj baze podataka. Klijenti koriste web
			preglednik kako bi pristupili web aplikaciji. Sustav je baziran na arhitekturi ”klijent – poslužitelj”, a
			komunikacija između računala korisnika (klijent, zaposlenik, vlasnik, administrator) i poslužitelja odvija se preko HTTP veze. }

			\begin{figure}[H]
				\includegraphics[width=\textwidth]{slike/Dijagram razmjestaja}
				\centering
				\caption{Dijagram razmještaja}
				\label{fig:razmijestaja}
			\end{figure}

			\eject 
		
		\section{Upute za puštanje u pogon}
		
            {Kako se aplikacija sastoji od dva distinktivna dijela, backend-a i frontend-a, svaki dio funkcionira samo za sebe te se zasebno pušta u pogon.}

			\subsection{Backend}

                {Backend se sastoji od koda u obliku gradle projekta u rađenom u Spring boot okruženju. Kako Spring boot samostalno radi tablice i ažurira relacije u bazi podataka, potrebno je samo na poslužitelju na kojem će se nalaziti kompajliran pokrenut kod stvoriti bazu podataka s nazivom \textit{cdb}.}
                
                {Za početak je potrebno otvoriti virtualni stroj s ubuntu serverom na poslužitelju po izboru nakon čega je potrebno unesti sljedeće naredbe koristeći command prompt kako bi se ulogirali i postavili server:}

                \begin{packed_item}
        			\item {ssh root@ipv4AdresaServera}
        			\item {sudo apt update}
        			\item {sudo apt upgrade}
        			\item {sudo apt install fail2ban}
        			\item {sudo systemctl restart fail2ban.service}
        		\end{packed_item}

                {Sada kada je server postavljen, potrebno je instalirati postgresql i podignuti bazu naziva cdb:}

                \begin{packed_item}
        			\item {sudo apt-get install postgresql postgresql-contrib}
        			\item {sudo systemctl start postgresql.service}
        			\item {sudo systemctl enable postgresql.service}
        			\item {sudo -u postgres psql}
        			\item {\textbackslash password postgres}
        			\item {CREATE DATABASE cdb}
        			\item {\textbackslash c cdb}
        			\item {\textbackslash q}
        		\end{packed_item}

                {Nakon što je baza podataka na serveru stvorena, potrebno je kompajlirati kod za backend aplikacije te ga tako kompajliranog prebaciti na server. Sljedeće naredbe potrebno je izvršiti u matičnog folderu koda preuzetnog na vlastito računalo:}

                \begin{packed_item}
        			\item {cd Backend}
        			\item {gradle bootJar}
        			\item {scp build/libs/backend-0.0.1-SNAPSHOT.jar root@ipv4AdresaServera:/var/www}
        		\end{packed_item}

                {Nakon što je kompajlirani kod na serveru, potrebno je instalirati javu kako bi ga mogli pokrenuti:}

                \begin{packed_item}
        			\item {sudo apt install openjdk-17-jdk}
        		\end{packed_item}

                {Kako bi osigurali da aplikacija radi i kada na serveru ne postoji ulogirani korisnik, potrebno je kompajliran kod pokrenuti kao servis:}

                \begin{packed_item}
        			\item {cd /usr/lib/systemd/system}
        			\item {nano runSpringServer.service}
        			\item \textit{Sljedeći kod upiši u novootvoreni text editor:}
        			\item {[Unit]}
                    \item {Description=webserver Daemon}
                    \item {[Service]}
                    \item {ExecStart=/usr/bin/java -jar /var/www/backend-0.0.1-SNAPSHOT.jar}
                    \item {User=root}
                    \item {[Install]}
                    \item {WantedBy=multi-user.target}
        			\item \textit{Zatvori text editor koristeći kombinacije tipki CTRL+S pa CTRL+X.}
        			\item {sudo systemctl start runSpringServer.service}
        			\item {sudo systemctl enable runSpringServer.service}
        		\end{packed_item}

            \subsection{Frontend}

                {Frontend je aplikacija rađena u React frameworku za čije je pokretanje i izgradnju potreban \textit{Node.js} koji je moguće preuzeti s adrese \textit{https://nodejs.org/en/download/}. Nakon instalacije \textit{Node.js}-a na vlastito računalo, u matičnom folderu koda preuzetog na vlastito računalo potrebno je izvršiti sljedeće naredbe kako bi kompajlirali frontend dio aplikacije:}

                \begin{packed_item}
        			\item {cd Frontend}
        			\item {npm install}
        			\item {npm run build}
        		\end{packed_item}

              {Nakon izvršenja ovih naredbi, u mapi Frontend će se stvoriti nova mapa naziva \textit{Build}. Ovu mapa sadrži kompajlirani kod Frontend aplikacije koji će biti potrebno prenesti na poslužitelj.}\vspace{0.3cm}

              {Kako bi korisnici kroz preglednik mogli pristupiti našoj aplikaciji potrebno ju je servirati na poslužitelju. Poslužitelj s kojeg će biti servirana naša aplikacija zove se \textit{Netlify}. Prvo je potrebno otići na adresu \textit{https://app.netlify.com/} i ulogirati se s već postojećim Github računom. Nakon toga potrebno je odabrati opciju \textit{Add new site} pa \textit{Deploy manually}. Prateći ove upute biti će ponuđen ekran na koji je potrebno ispustiti \textit{Build} mapu napravljenu u prethodnom koraku.}

              {Nakon nekoliko minuta Frontend aplikacija će se poslužiti te će nam Netlify dati IP adresu preko koje možemo pristupiti svojoj aplikaciji \textit{(npr. https://darling-lokum-558ee5.netlify.app)}.}\vspace{0.3cm}
              
              {Po želji, preteći Netlify dokumentaciju moguće je promijeniti IP adresu aplikacije kao i mnoge druge postavke koje Netlify nudi, no koji nisu potrebne za funkcioniranje aplikacije.}
			\eject 